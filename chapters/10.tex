\section{HTTP Method Request}
\subsection{Pengertian HTTP Method Request}
\subsection {Mekanisme HTTP Method Request}
\subsection {Contoh URL HTTP Get Method}
\subsection {Mendapatkan Parameter GET Python Flask}
    \begin{enumerate}
        \item Pengenalan Python
        
        Python adalah salah satu bahasa pemograman tingkat tinggi yang bersifat interpreter, interactive, objectoriented, dan dapat beroperasi hampir di semua platform: Mac, Linux, dan Windows. Python termasuk bahasa pemograman yang mudah dipelajari karena sintaks yang jelas, dapat dikombinasikan dengan penggunaan modulmodul siap pakai, dan struktur data tingkat tinggi yang efisien \cite{kadir2005dasar}. Python mendukung multi paradigma pemrograman, utamanya; namu tidak dibatasi; pada pemrograman berorientasi objek, pemrograman imperatif, dan pemrograman fungsional. Salah satu fitur yang tersedia pada python adalah sebagai bahasa pemrograman dinamis yang dilengkapi dengan manajemen memori otomatis.
        
        Seperti halnya pada bahasa pemrogrman dinamis lainnya, python umumnya digunakan sebagai bahasa skrip meski pada praktiknya penggunaan bahasa ini lebih luas mencakup konteks pemnfaatan yang umumnya tidak dilakukan dengan menggunakan bahasa skrip. Python dapat digunakan untuk berbagai keperluan pengembangan perangkat lunak dan dapat berjalan di berbagai platform sistem operasi.

        \item Pengenalan Flask
    \end{enumerate}

\section{Macam-Macam Penanganan Error Proyek}

\subsection{Penanganan Error pada Python dan Flask}
\begin{enumerate}
  \item Contoh Kasus 1 : Penerapan fungsi sederhana yang dieksekusi dicommand prompt. Contoh pemanggilan fungsi apabila dieksekusi di CMD, seperti gambar \ref{fig:contohsederhana}

  \begin{figure}[!ht]
        \centerline{\includegraphics[width=0.70\textwidth]{figures/10/contohsederhana.jpg}}
	    \caption{Fungsi Sederhana}
	    \label{fig:contohsederhana}
  \end{figure}


ini adalah contoh untuk pengeksekusian file python yang berupa gunsi yang telah dibuat. Berikut langkah-langkahnya :
    \begin{itemize}
        \item Petama-tama masukkan kedalam directory tempat anda menyimpan file yang telah anda buat.
        \item kemudian pada directory tersebut ketik python
        \item Setelah masuk kedalam python silahkan masukkan file python basreng
    \end{itemize}

  \item Contoh kasus 2 : Kode pembawa sinyal gelombang otak (NeuroSky Mindwave EEG). Kodenya seperti contoh \ref{lst:coba}, silahkan tutorialnya diikuti terlebih dahulu.
\lstinputlisting[caption=Contoh kode untuk membaca sinyal gelombang otak,label={lst:coba}]{src/10/coba.py}
\end{enumerate}

